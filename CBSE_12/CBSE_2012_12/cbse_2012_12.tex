\let\negmedspace\undefined
\let\negthickspace\undefined
\documentclass[journal,12pt,twocolumn]{IEEEtran}
\usepackage{gensymb}
\usepackage{amssymb}
\usepackage[cmex10]{amsmath}
\usepackage{amsthm}
\usepackage[export]{adjustbox}
\usepackage{bm}
\usepackage{longtable}
\usepackage{enumitem}
\usepackage{mathtools}
\usepackage[breaklinks=true]{hyperref}
\usepackage{listings}
\usepackage{color}                                            %%
\usepackage{array}                                            %%
\usepackage{longtable}                                        %%
\usepackage{calc}                                             %%
\usepackage{multirow}                                         %%
\usepackage{hhline}                                           %%
\usepackage{ifthen}                                           %%
\usepackage{lscape}     
\usepackage{multicol}
\usepackage{tfrupee}
% \usepackage{enumerate}
\DeclareMathOperator*{\Res}{Res}
\renewcommand\thesection{\arabic{section}}
\renewcommand\thesubsection{\thesection.\arabic{subsection}}
\renewcommand\thesubsubsection{\thesubsection.\arabic{subsubsection}}
\renewcommand\thesectiondis{\arabic{section}}
\renewcommand\thesubsectiondis{\thesectiondis.\arabic{subsection}}
\renewcommand\thesubsubsectiondis{\thesubsectiondis.\arabic{subsubsection}}
\newcommand{\uvec}[1]{\boldsymbol{\hat{\textbf{#1}}}}
\hyphenation{op-tical net-works semi-conduc-tor}
\def\inputGnumericTable{}  %%

\lstset{
frame=single, 
breaklines=true,
columns=fullflexible
}
\begin{document}
\newtheorem{theorem}{Theorem}[section]
\newtheorem{problem}{Problem}
\newtheorem{proposition}{Proposition}[section]
\newtheorem{lemma}{Lemma}[section]
\newtheorem{corollary}[theorem]{Corollary}
\newtheorem{example}{Example}[section]
\newtheorem{definition}[problem]{Definition}
\newcommand{\BEQA}{\begin{eqnarray}}
\newcommand{\EEQA}{\end{eqnarray}}
\newcommand{\define}{\stackrel{\triangle}{=}}
\bibliographystyle{IEEEtran}
\providecommand{\mbf}{\mathbf}
\providecommand{\pr}[1]{\ensuremath{\Pr\left(#1\right)}}
\providecommand{\qfunc}[1]{\ensuremath{Q\left(#1\right)}}
\providecommand{\sbrak}[1]{\ensuremath{{}\left[#1\right]}}
\providecommand{\lsbrak}[1]{\ensuremath{{}\left[#1\right.}}
\providecommand{\rsbrak}[1]{\ensuremath{{}\left.#1\right]}}
\providecommand{\brak}[1]{\ensuremath{\left(#1\right)}}
\providecommand{\lbrak}[1]{\ensuremath{\left(#1\right.}}
\providecommand{\rbrak}[1]{\ensuremath{\left.#1\right)}}
\providecommand{\cbrak}[1]{\ensuremath{\left\{#1\right\}}}
\providecommand{\lcbrak}[1]{\ensuremath{\left\{#1\right.}}
\providecommand{\rcbrak}[1]{\ensuremath{\left.#1\right\}}}
\theoremstyle{remark}
\newtheorem{rem}{Remark}
\newcommand{\sgn}{\mathop{\mathrm{sgn}}}
\providecommand{\abs}[1]{\left\vert#1\right\vert}
\providecommand{\res}[1]{\Res\displaylimits_{#1}} 
\providecommand{\norm}[1]{\left\lVert#1\right\rVert}
%\providecommand{\norm}[1]{\lVert#1\rVert}
\providecommand{\mtx}[1]{\mathbf{#1}}
\providecommand{\mean}[1]{E\left[ #1 \right]}
\providecommand{\fourier}{\overset{\mathcal{F}}{ \rightleftharpoons}}
%\providecommand{\hilbert}{\overset{\mathcal{H}}{ \rightleftharpoons}}
\providecommand{\system}{\overset{\mathcal{H}}{ \longleftrightarrow}}
	%\newcommand{\solution}[2]{\textbf{Solution:}{#1}}
\newcommand{\solution}{\noindent \textbf{Solution: }}
\newcommand{\cosec}{\,\text{cosec}\,}
\providecommand{\dec}[2]{\ensuremath{\overset{#1}{\underset{#2}{\gtrless}}}}
\newcommand{\myvec}[1]{\ensuremath{\begin{pmatrix}#1\end{pmatrix}}}
\newcommand{\mydet}[1]{\ensuremath{\begin{vmatrix}#1\end{vmatrix}}}
\numberwithin{equation}{subsection}
\makeatletter
\@addtoreset{figure}{problem}
\makeatother
\let\StandardTheFigure\thefigure
\let\vec\mathbf
\renewcommand{\thefigure}{\theproblem}
\def\putbox#1#2#3{\makebox[0in][l]{\makebox[#1][l]{}\raisebox{\baselineskip}[0in][0in]{\raisebox{#2}[0in][0in]{#3}}}}
     \def\rightbox#1{\makebox[0in][r]{#1}}
     \def\centbox#1{\makebox[0in]{#1}}
     \def\topbox#1{\raisebox{-\baselineskip}[0in][0in]{#1}}
     \def\midbox#1{\raisebox{-0.5\baselineskip}[0in][0in]{#1}}
\vspace{3cm}
\title{
	12th CBSE MATHEMATICS
}
\author{ 2012-13
	\thanks{}
	
}
\maketitle
\newpage
\bigskip
\renewcommand{\thefigure}{\theenumi}
\renewcommand{\thetable}{\theenumi}
\section{Section A}
\renewcommand{\theequation}{\theenumi}
\begin{enumerate}[label=\thesection.\arabic*.,ref=\thesection.\theenumi]
\numberwithin{equation}{enumi}
\item If the binary operation $*$ on the set $Z$ of integers is defined by $a*b=a+b-5$, then write the identity element for the operation $*$ in $Z$.\\
\item Write the values of $\cot(tan^{-1}a+\cos^{-1}a)$ \\
\item If $A$ is a square matrix such that $A^2 = A$, then write the value of $(I+A)^2-3A$\\
\item If $x \begin{bmatrix} 2 \\ 3 \end{bmatrix}	+ y \begin{bmatrix} -1 \\ 1 \end{bmatrix} = \begin{bmatrix} 10 \\ 5 \end{bmatrix}$, write the value of $x$.\\	
\item Write the following determinant : $\begin{vmatrix} 102 & 18 & 36 \\ 1 & 3 & 4\\ 17 & 3 & 6 \end{vmatrix}$\\
\\
\item If $\int(\frac{x-1}{x^2})e^{x}dx= f(x)e^{x}+c$, then write the value of $f(x)$\\
\item If $\int_{a}^{0}3x^{2}dx=8$, write the value of $'a'$.\\
\item Write the value of $(\hat{i} \ $x$ \ \hat{j}).\hat{k}+(\hat{j} \ $x$ \ \hat{k}).\hat{i}$ \\
\item Write the value of the area of the parallelogram determined by the vectors $2\hat{i}$ and $3\hat{j}$\\
\item Write the direction cosines of a line parallel to z-axis\\
\end{enumerate}
\section{Section B}
\renewcommand{\theequation}{\theenumi}
\begin{enumerate}[label=\thesection.\arabic*.,ref=\thesection.\theenumi]
\numberwithin{equation}{enumi}
\item If  $f(x) = \frac{4x+3}{6x-4}$, x $\neq \frac{2}{3}$, show that $fof(x) = x$ all $x \neq \frac{2}{3}$. What is the inverse of f? \\
\item Prove that: \\
 $\sin^{-1}(\frac{63}{65})=\sin^{-1}(\frac{5}{13})+\cos^{-1}(\frac{3}{5})$
 \begin{align}
    \centering \vec{OR}\nonumber
\end{align}
Solve for x:\\
$2 \tan^{-1}(\sin x)=\tan^{-1}(2 \sec x), x \neq \frac{\pi}{2}$\\
\item Using properties of determinants, prove that\\ $\begin{vmatrix} a & a+b & a+b+c \\ 2a & 3a+2b & 4a+3b+2c \\ 3a & 6a+3b & 10a+6b+3c \end{vmatrix}= a^{3}$\\
\item If $x^{m} \ y^{n}=(x+y)^{m+n},$ prove that $\frac{dy}{dx}=\frac{y}{x}$\\
\item If $y = e^{a \cos^{-1}x}, -1 \leq x < 1,$ show that \\ $(1-x^{2})\frac{d^{2}y}{dx^{2}}- x \frac{dy}{dx}-a^{2}y=0$ \\
\begin{align}
    \centering \vec{OR}\nonumber
\end{align}
If $x\sqrt{1+y}+y\sqrt{1+x}=0$,$-1 < x < 1$,$x \neq y$, then prove that \\ $\frac{dy}{dx}=-\frac{1}{(1+x^{2})}$\\
\item Show that $y = \log(1+x)-\frac{2x}{2+x}$,$x>-1$, is an increasing function of $x$ throughout its domain.\\
\begin{align}
    \centering \vec{OR}\nonumber
\end{align}
\item Find the equation of the normal at the point $(am^{2}, am^{3})$ for the curve $ay^{2} = x^{3}$ \\
\item Evaluate\\
$\int x^{2}tan^{-1}x\ dx$
\begin{align}
    \centering \vec{OR}\nonumber
\end{align}
Evaluate \\
$\int \frac{3x-1}{(x+2)^{2}}dx$ \\
\item Solve the following differential equation : \\
$[\frac{e^{-2\sqrt{x}}}{\sqrt{x}}-\frac{y}{\sqrt{x}}]\frac{dx}{dy}= 1$, $x\neq 0$\\
\item Solve the following differential equation : \\
$3\ e^{x}\tan y \ dx + (2-e^{x})\sec^{2}y\ dy =0$, given that when $x=0$, $y=\frac{\pi}{4}$\\
\item If $\overrightarrow{\alpha} = 3\hat{i}+ 4\hat{j}+5\hat{k}$ and $\overrightarrow{\beta}=2\hat{i}+\hat{j}-4\hat{k}$, then express $\overrightarrow{\beta}$ in the form $\overrightarrow{\beta}=\overrightarrow{\beta_1}+\overrightarrow{\beta_2}$, where $\overrightarrow{\beta_1}$ is parallel of $\overrightarrow{\alpha}$ and $\overrightarrow{\beta_2}$ is perpendicular to $\overrightarrow{\alpha}$.\\
\item Find the vector and cartesian equations of the line passing through the point $P(1,\ 2,\ 3)$ and parallel to the planes $\overrightarrow{r}$. $(\hat{i}-\hat{j}+2\hat{k}=5$ and $\overrightarrow{r}$ . $(3\hat{i}+\hat{j}+\hat{k})=6$.\\
\item A pair of dice is thrown 4 times. If getting a doublet is considered a success, find the probability distribution of the number of successes and hence find its mean.\\ 
\end{enumerate}
\section{Section C}
\renewcommand{\theequation}{\theenumi}
\begin{enumerate}[label=\thesection.\arabic*.,ref=\thesection.\theenumi]
\numberwithin{equation}{enumi}
\item Using matrices, solve the following system of equations:\\
$x-y+z=4;\ 2x+y-3z=0; \ x+y+z=2 $
\begin{align}
    \centering \vec{OR}\nonumber
\end{align}
If $A^{-1}=\begin{bmatrix}3&-1&1\\-15&-6&-5\\5&-2&2\end{bmatrix}$ and $B=\begin{bmatrix}1&2&-1\\-1&3&0\\0&-2&1\end{bmatrix}$, find $(AB)^{-1}$.\\
\item Show that the altitude of the right circular cone of maximum volume that can be inscribed in a sphere of radius is $\frac{4R}{3}$.\\
\item Find the area of the region in the first quadrant enclosed by x-axis, the line $x = \sqrt{3}\ y$ and the circle $x^{2}+y^{2}=4$. \\
\item Evaluate:\\
$\int_{1}^{3}(x^{2}+x)dx$ as a limit of a sum.
\begin{align}
    \centering \vec{OR}\nonumber
\end{align}
Evaluate: \\
$\int_{0}^{\pi/4}\frac{\cos^{2}}{\cos^{2}x+4\sin^{2}x}dx$\\
\item Find the vector equation of the plane passing through the points (2, 1, -1) and (-1, 3, 4) and perpendicular to the plane $x-2y+4z=10$. Also show that the plane thus obtained contains the lines $\overrightarrow{r}=\hat{i}+3\hat{j}+4\hat{k}+\lambda(3\hat{i}-2\hat{j}-5\hat{k}).$\\
\item A company produces soft drinks that has a contract which requires that a minimum of 80 units of the chemical A and 60 units of the chemical B go into each bottle of the drink. The chemicals are available in prepared mix packets from two different suppliers. Supplier S had a packet of mix of 4 units of A and 2 units of B that costs $\rupee 10$. The supplier T has a packet of mix of 1 unit of A and 1 unit of B that costs $\rupee 4$. How many packets of mixes from S and T should the company purchase to honour the contract requirement and yet minimize cost ? Make a LPP and solve graphically.\\
\item In a certain college, $4\%$ of boys and $1\%$ of girls are taller than 1.75 metres. Furthermore, $60\%$ of the students in the college are girls. A student is selected at random from the college and is found to be taller than 1.75 metres. Find the probability that the selected student is a girl.
 \end{enumerate}
\end{document}
